\chapter{State Space Analysis of the CPN Web Socket Protocol}
\label{chap:statespace}

One of the advantages of Coloured Petri Nets is the ability to conduct state
space analysis, which can be used to obtain information about the
behavioural properties of a CPN model, and which can be used to locate errors
and increase confidence in the correctness of the CPN model.

\section{State Spaces}
A state space is a directed graph where each node represents a reachable marking
(a state) and each arc represents an occurring binding element (a transition
firing with values bound to the variables of the transition). CPN Tools
by default generates the state space in breadth-first order. 

TODO: Figur av SS initiell modell, med forklaring

Once generated, the state space can be visualised directly in CPN Tools.
Starting with the node for the initial state, one can pick a node and show all
nodes that are reachable from it, and in this way explore the state space
manually. This can be very tedious and unmanageable for complex state spaces,
though, and instead it is usually better to use queries to automate the analysis
based on state spaces.

	\subsection{Strongly Connected Component graph}
	In graph theory, a strongly connected component (SCC) of a graph is a maximal
	subgraph where all nodes are reachable from each other. An SCC graph has a node
	for each SCC of the graph, connected by arcs determined by the arcs in the
	underlying graph between nodes that belong to different SCCs. An
	SCC graph is acyclic, and an SCC is said to be trivial if it consists of only
	one node from the underlying graph.
	
	By calculating the SCC graph of the state space, some of the further
	analysis becomes simpler and faster, such as determining reachability,
	cyclic behaviour, and checking so-called home and liveness properties. 

	\subsection{Application of State Spaces}
		The biggest drawback of state space analysis is the size a state spaces may
		become very large. The number of nodes and arcs often grows exponentially
		with the size of the model configuration.
		This is also known as the state explosion problem.
		
		This can be remedied by picking smaller configurations that encapsulate
		different parts of the system. This was necessary with the WebSocket Protocol
		model, as the complete state space took too long to generate with our
		original configuration. \com{illustrasjon}
		We started by removing all messages to be sent. This means the only thing that
		should happen during simulation is the opening handshake. This configuration
		is used to explain the State Space Report in the next section.
		After this, we gradually added different types of messages to the client
		and/or server applications. These configurations will be discussed at the end
		of the chapter.
		
		Another aspect that must be considered prior to state space analysis is
		situations where an unlimited number of tokens can be generated, thus making
		the state space infinite. This can be remedied by modifying the model to
		limit the number of simultaneous tokens in the offending place.
		
		Additionally, a model that incorporates random values is not always suited
		for computing a state space. The generated state space depends on the random
		values chosen, so the state space generator needs to be able to
		deterministically bind values to arc expression variables.
				
		For small color sets (generally defined as discrete sets usually with less
		than 100 possible values), binding of random values in arc expressions can
		occur in two ways: 
		\begin{enumerate}
		\item By calling ran() on the colorset. The ran() function  over the
		color set, but since is non-deterministic, it isn't suited for state space
		generation.
		\item By using a free variable ranging over a color set in the arc expression.
		A free variable is a variable that does not get assigned a value in an expression. It will 
		bind to a value picked at random from the color set during simulation just
		like the ran() function, but also lets the state space generator pick each of 
		the possible bindings from the values available in the colorset, and thus
		generate all possible successive states. 
		\end{enumerate}
		
		For arc expressions that use type 1, it is usually possible to change or
		adapt it into type 2.
		
		Color sets that use values from a large or unbounded range, or from continuous
		ranges like floating point numbers, are considered large color sets, and using
		random values from such color sets can make it impossible (or impractical)
		to generate a complete state space. It can be worked around by
		instead using small color sets as described above. The CPN Tools manual has
		examples on how to do this.
		
		If these issues are not taken into account, a complete state space can not be
		achieved, since it's impossible for the state space genrator to make sure all
		possible values have been considered, and occurrence sequences might diverge
		if the same occurrence can happen in different orders but with different
		random values.
		
		For the WebSocket Protocol model, this was a problem for the masking
		key in WebSocket frames, which is supposed to be a random 4-byte string,
		giving $2^{32}$ or almost 4.3 billion possible values.
		To generate state spaces for this model, the randomisation function used was
		simply changed to always return four zeros. This is a reasonable abstraction
		since the specific value of the masking key does not affect the operation of
		the protocol. The result is shown in \lstref{fixed_masking_key}, with the old
		code commented out.
		
	\begin{lstlisting}[label=lst:fixed_masking_key,caption=Fixed masking key] 
	fun	randMask() = Mask([ 
		Byte(0),Byte(0),Byte(0),Byte(0)
		(* BYTE.ran(), BYTE.ran(), BYTE.ran(), BYTE.ran() *)
	]);
	\end{lstlisting}

\section{State Space Report}
Once a partial or complete state space has been generated, CPN Tools lets the
user save a state space report as a textual document. The report is organised
into parts that each describe different behavioural properties of the state
space.

To explain each section of the state space report, a simple report for the
WebSocket protocol has been generated, in a configuration where no messages are
set to be sent. Thus, the only thing that will happen is that a connection will
be established. Later in the chapter we will consider more elaborate
configurations of the WebSocket protocol.
	
	\subsection{Statistics}
	The first section of the report describes general statistics about the state
	space.
	\begin{lstlisting}
  State Space
     Nodes:  17
     Arcs:   16
     Secs:   0
     Status: Full

  Scc Graph
     Nodes:  17
     Arcs:   16
     Secs:   0

	\end{lstlisting}
	This state space has 17 possible markings, with 16 enabled transition
	occurrences connecting them. There is one more node than there are arcs, which
	means this graph is a tree.
	
	The |Secs| field shows that it took less than one second to calculate this
	state space, while the |Status| field tells whether the report is generated
	from a partial or full state space. In this case the state space is fully
	generated.
	
	We also see that the SCC Graph has the same number of nodes and arcs, meaning
	that there are no cycles in the state space (although this was already known
	from the fact that it is a tree).
	
	\subsection{Boundedness Properties}
	The second section describes the minimum and maximum number of tokens for
	each place in the model, as well as the actual tokens these places can have.
	The text has been reformatted and truncated (indicated by [\ldots]) for
	readability.
	\begin{lstlisting}[language={}]
  Best Integer Bounds
                             Upper      Lower
     ClientApplication
          Active_Connection
                             1          0
          Conn_Result
                             1          0
          Connection_failed
                             0          0
          Messages_received
                             1          1
          Messages_to_be_sent
                             0          0
     [...]

	\end{lstlisting}
	
	Many places show a lower and upper bound of 1. This shows a weakness
	in the approach of using lists to facilitate ordered processing of tokens: We
	cannot see the actual number of tokens that are in the place, because
	technically there is just a list there. However, it quickly lets us know if
	something is wrong as well, since any values other than 0 or 1 here indicate a
	problem. 
	
	In fact, an error in the model was discovered this way, in the Unwrap and
	Receive module, where the pong reply was creating a new list instead of
	appending to the old one in outgoing messages. This caused the Client Outgoing
	Messages place to have 2 tokens at once. \figref{pong_fix} shows the location
	of the error before and after fixing it.
	\figDbl[0.35]{SSA_UnwrapAndReceive_Pong_before.eps}{Before}{SSA_UnwrapAndReceive_Pong_after.eps}{After}{Fixing
	Pong reply}{pong_fix}
	
	\begin{lstlisting}[language={}]
  Best Upper Multi-set Bounds
     ClientApplication
          Active_Connection
                         1`()
          Conn_Result
                         1`success
          Connection_failed
                         empty
          Messages_received
                         1`[]
          Messages_to_be_sent
                         empty
     [...]
     ClientWebSocket
          Connection_status
                         1`CONN_OPEN
     [...]
     ServerWebSocket
          Connection_Status
                         1`CONN_OPEN
     [...]

  Best Lower Multi-set Bounds
     ClientApplication
          Active_Connection
                         empty
          Conn_Result
                         empty
          Connection_failed
                         empty
          Messages_received
                         1`[]
          Messages_to_be_sent
                         empty
     [...]
     ClientWebSocket
          Connection_status
                         empty
     [...]
     ServerWebSocket
          Connection_Status
                         empty
     [...]
	\end{lstlisting}
	
	Apart from that, we see that both the client and the server has an open
	connection at some point, as the |Connection_status| place in the
	|ClientWebSocket| and |ServerWebSocket| modules have both had a |CONN_OPEN|
	token.
	
	\subsection{Home Properties}
	This section shows all home markings. A home marking is a marking that can
	always be reached no matter where we are in the state space. 
	\begin{lstlisting}[language={}]
  Home Markings
     [17]
	\end{lstlisting}
	We see that there is one such marking defined by node 17. From earlier we know
	that the state space is a tree, and if this node is always reachable it must be a leaf
	and all the other nodes must be in a chain. This tells us that there is only
	one possible sequence of transitions to establish a connection. We can then
	confidently say that the model works correctly with this configuration.
	
	\subsection{Liveness Properties}
	This section describes liveliness of the state space. Some of the transitions
	have been omitted for readability.
	
	\begin{lstlisting}[language={}]
  Dead Markings
     [17]


  Dead Transition Instances
     ClientApplication'Fail 1
     ClientApplication'Receive_data 1
     ClientApplication'Send_data 1
     ClientWebSocket'Filter_messages 1
     .....

  Live Transition Instances
     None
	\end{lstlisting}
	
	A dead marking is a marking from where no other markings can be reached.
	\com{from where? for which? formuler} In other words, there are no transitions
	for which there are enabled bindings, and the system is effectivley stopped.
	For our example, we have a single dead marking, and it is the same as our home
	marking, confirming that this is a leaf node in the tree.
	
	We also get a listing of dead transition instances, which are transitions that
	never have any enabled bindings in a reachable marking and are thus never
	fired. This can be useful to detect problems with a model, but in this example
	it is expected for many of the transitions, since we are not sending any kind
	of messages in the configuration considered.
	
	Last, there are live transition instances. A transition is live if we from any
	reachable marking can find an occurrence sequence containing the transition.
	Our example has no such transition, which follows trivially from the fact that
	there is a dead marking.

	The state space report also contains fairness properties, but this does
	not apply to our model since it contains no cycles. We will not go into detail
	about this, and instead refer to \cite{cpn_book} chapter 7 for more
	information.
	
\section{TODO:} 
Skrive om resten av analysene.
