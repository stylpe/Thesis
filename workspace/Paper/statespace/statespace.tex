\chapter{State Space Analysis of the WebSocket Protocol}
\label{chap:statespace}

One of the advantages of Coloured Petri Nets is the ability to conduct state
space analysis, which can be used to obtain information about the
behavioural properties, as well as to locate errors
and increase confidence in the correctness of a CPN model, and in our case the
communication protocol being modeled.

\section{State Spaces}
A state space is a directed graph where each node represents a reachable marking
(a state) and each arc represents an occurring binding element (a transition
firing with values bound to the variables of the transition). CPN Tools will
by default calculate the state space in breadth-first order. 

Once generated, the state space graph can be visualised directly in CPN Tools.
For example, starting with the node for the initial state, one can pick a node
and display all nodes that are reachable from it, and in this way explore the state space
manually. This can be very tedious and unmanageable for complex state spaces,
though, and instead it is usually better to use queries written in CPN ML to
automate the analysis based on state spaces.

	\subsection{Strongly Connected Component graph}
	In graph theory, a strongly connected component (SCC) of a graph is a maximal
	subgraph where all nodes are reachable from each other. An SCC graph has a node
	for each SCC of the graph, connected by arcs determined by the arcs in the
	underlying graph between nodes that belong to different SCCs. An
	SCC graph is acyclic, and an SCC is said to be trivial if it consists of only
	one node from the underlying graph.
	
	By calculating the SCC graph of the state space, the analysis of certain
	properties becomes simpler and faster, such as determining reachability,
	determining cyclic behaviour, and checking so-called home and liveness
	properties.

	\subsection{Application of State Spaces}
		The biggest drawback of state space analysis is the size of the state space
		may become very large. The number of nodes and arcs often grows exponentially
		with the size of the model configuration.
		This is also known as the state explosion problem.
		
		\figDbl{ss/ClientApplicationNoMessagesCropped.eps}{Client}
		{ss/ServerApplicationNoMessagesCropped.eps}{Server}{Configuration:
		No messages}{ss_no_messages}
	
		This can be remedied by picking smaller configurations that encapsulate
		different parts of the system. This was necessary with the WebSocket Protocol
		model, as the complete state space took too long to generate with our
		original configuration (see \figref{client_app} and \figref{server_app}).
		We started by removing all messages to be sent (shown in
		\figref{ss_no_messages}).
		This means the only thing that should happen during simulation is the opening
		handshake. This configuration is used to explain the State Space Report in
		the next section.
		After this, we gradually added different types of messages to the client
		and/or server applications. These configurations will be discussed at the end
		of the chapter.
		
		Another aspect that must be considered prior to state space analysis is
		situations where an unlimited number of tokens can be generated on a place,
		thus making the state space infinite. This can be remedied by modifying the model to
		limit the number of simultaneous tokens in the offending place.
		
		Additionally, a model that incorporates random values is not always suited
		for computing a state space. The generated state space depends on the random
		values chosen, so the state space generator needs to be able to
		deterministically bind values to arc expression variables.
				
		For small colour sets (generally defined as discrete types with less
		than 100 possible values), binding of random values in arc expressions can
		occur in two ways: 
		\begin{enumerate}
		\item By calling ran() on the colourset. The ran() function picks a value
		ranging over the colour set, but since the choice is non-deterministic,
		this construct is not suited for state space generation.
		\item By using a free variable ranging over a colour set in the arc expression.
		A free variable is a variable that does not get assigned a value in an expression. It will 
		bind to a value picked at random from the colour set during simulation just
		like the ran() function, but also lets the state space generator pick each of 
		the possible bindings from the values available in the colourset, and thus
		generate all possible successive states. 
		\end{enumerate}
		
		For arc expressions that use type 1, it is usually possible to change or
		adapt it into type 2.
		
		Colour sets that use values from a large or unbounded range, or from continuous
		ranges like floating point numbers, are considered large colour sets, and using
		random values from such colour sets can make it impossible (or impractical)
		to generate a complete state space. It can be worked around by
		instead using small colour sets as described above. The CPN Tools manual has
		examples on how to do this.
		
		If these issues are not taken into account, a complete state space can not be
		obtained, since it is impossible for the state space genrator of CPN Tools to
		make sure that all possible values have been considered, and occurrence
		sequences might diverge if the same occurrence can happen in different orders but with different
		random values.
		
		For the WebSocket Protocol model, this was a problem for the masking
		key in WebSocket frames, which is supposed to be a random 4-byte string,
		giving $2^{32}$ or almost 4.3 billion possible values.
		To generate state spaces for this model, the randomisation function used was
		simply changed to always return four zeros. This is a reasonable abstraction
		since the specific value of the masking key does not affect the operation of
		the protocol. The result is shown in \lstref{fixed_masking_key}, with the old
		code commented out.
		
		\begin{lstlisting}[label=lst:fixed_masking_key,gobble=2,caption=Fixed masking
		key] 
		fun	randMask() = Mask([ 
			0,0,0,0
			(* BYTE.ran(), BYTE.ran(), BYTE.ran(), BYTE.ran() *)
		]);
		\end{lstlisting}
	
	\subsection{Visualisation}
	\fig[0.4]{ss/vis_NoMessages.eps}{State space for
	configuration with no messages}{ss_vis_nomessage} CPN Tools can visualise a
	state space once it has been calculated.
	\figref{ss_vis_nomessage} shows the state space for the no messages
	configuration. Rounded squares represent markings, and arcs represent
	transition occurences. Clicking on the small triangle in the node will display
	a node descriptor which shows the marking that is associated with the node.
	Similarly, clicking on a state space arc will display an arc descriptor which
	describes the binding element associated with the arc. 

\section{State Space Report}
Once a partial or complete state space has been generated, CPN Tools lets the
user save a state space report as a textual document. The report is organised
into parts that each describe different behavioural properties of the CPN model.

To explain each section of the state space report, a simple report for the
WebSocket protocol has been generated, in a configuration where no messages are
set to be sent. Thus, the only thing that will happen is that a connection will
be established. Later in the chapter we will consider more elaborate
configurations of the WebSocket protocol.
	
	\subsection{Statistics}
	The first section of the report is shown on \lstref{ssa_statistics} and
	describes general statistics about the state space.
	\begin{lstlisting}[language={},float=h,label=lst:ssa_statistics,
	caption=State Space Report: Statistics]
  State Space
     Nodes:  17
     Arcs:   16
     Secs:   0
     Status: Full

  Scc Graph
     Nodes:  17
     Arcs:   16
     Secs:   0

	\end{lstlisting}
	This state space has 17 possible markings, with 16 enabled transition
	occurrences connecting them. There is one more node than there are arcs, which
	means this graph is a tree (in fact it consists of just a single path as was
	shown in \figref{ss_vis_nomessage})
	
	The |Secs| field shows that it took less than one second to calculate this
	state space, while the |Status| field specifies whether the report is generated
	from a partial or full state space. In this case, the state space is fully
	generated.
	
	We also see that the SCC Graph has the same number of nodes and arcs, meaning
	that there are no cycles in the state space (although this was already known
	from the fact that it is a tree).
	
	\subsection{Boundedness Properties}
	The second section of the state space report , shown in \lstref{ssa_bib},
	describes the minimum and maximum number of tokens for each place in the model, as well as the actual tokens these places can have.
	The text has been reformatted and truncated (indicated by [\ldots]) for
	readability.
	\begin{lstlisting}[language={},float=h,label=lst:ssa_bib,
	caption=State Space Report: Best Integer Bounds]
    Best Integer Bounds
                             Upper      Lower
     ClientApplication
          Active_Connection   1          0
          Conn_Result         1          0
          Connection_failed   0          0
          Messages_received   1          1
          Messages_to_be_sent 0          0
     [...]
	\end{lstlisting}
	
	Many places have a lower and upper bound of 1. This shows a weakness
	in the approach of using lists to facilitate ordered processing of tokens: We
	cannot see the actual number of tokens (i.e. the number of elements in the
	list) that are in each place, because technically there is just one token
	there: the list itself.
	However, it quickly lets us know if something is wrong as well, since any
	values other than 0 or 1 here would indicate a problem. 
	
	In fact, an error in the model was discovered this way, in the Unwrap and
	Receive module, where the Pong reply was creating a new list instead of
	appending to the old one in outgoing messages. This caused the \nodename{Client
	Outgoing Messages} place to have 2 tokens at once. \figref{pong_fix} shows the
	location of the error before and after fixing it.
	\figDbl[0.35]{SSA_UnwrapAndReceive_Pong_before.eps}{Before}{SSA_UnwrapAndReceive_Pong_after.eps}{After}{Problem
	with the Pong reply modeling}{pong_fix}
	
	\begin{lstlisting}[language={},tabsize=4,float,label=lst:ssa_bmsb,
	caption=State Space Report: Best Multi-set Bounds]
  Best Upper Multi-set Bounds
     ClientApplication
          Active_Connection		1`()
          Conn_Result			1`success
          Connection_failed		empty
          Messages_received		1`[]
          Messages_to_be_sent	empty
     [...]
     ClientWebSocket
          Connection_status		1`CONN_OPEN
     [...]
     ServerWebSocket
          Connection_Status		1`CONN_OPEN
     [...]

  Best Lower Multi-set Bounds
     ClientApplication
          Active_Connection		empty
          Conn_Result			empty
          Connection_failed		empty
          Messages_received		1`[]
          Messages_to_be_sent	empty
     [...]
     ClientWebSocket
          Connection_status		empty
     [...]
     ServerWebSocket
          Connection_Status		empty
     [...]
	\end{lstlisting}
	
	The Best Upper Multi-set Bounds, shown in \lstref{ssa_bmsb}, will show for each
	place a set of every token that exists in that place at some point in the state space. We see that both
	the client and the server has an open connection at some point, as the
	\nodename{Connection\_status} place in the \nodename{ClientWebSocket} and
	\nodename{ServerWebSocket} modules have both had a |CONN_OPEN| token.
	
	The Best Lower Multi-set Bounds is the opposite, and shows the smallest set of
	tokens that exists at any point in the state space. This is either the empty
	list |[]| or simply |empty| for all places.
	
	\subsection{Home Properties}
	This section of the state space report, shown in \lstref{ssa_home}, shows all
	home markings. A home marking is a marking that can always be reached from any
	other reachable state in the state space.
	
	\begin{lstlisting}[language={},float,label=lst:ssa_home,
	caption=State Space Report: Home Properties]
  Home Markings
     [17]
	\end{lstlisting}
	
	We see that there is one such marking defined by node 17. From earlier we know
	that the state space is a tree, and if this node is always reachable it must be a leaf
	and all the other nodes must be in a chain. This agrees with the visualisation
	shown earlier in \figref{ss_vis_nomessage} that there is only one possible
	sequence of transition occurrences to establish a connection. CPN Tools
	allows us to import the state from this node into the editor, and by manually
	inspecting the various markings of the model to verify the desired state of
	an open connection exists with no side effects, we can confidently say that the
	model works correctly with this configuration.
	
	\subsection{Liveness Properties}
	This section of the state space report, shown in \lstref{ssa_liveness},
	describes so-called liveness properties of the state space.
	Some of the transitions have been omitted for readability.
	
	\begin{lstlisting}[language={},float,label=lst:ssa_liveness,
	caption=State Space Report: Liveness Properties]
  Dead Markings
     [17]

  Dead Transition Instances
     ClientApplication'Fail 1
     ClientApplication'Receive_data 1
     ClientApplication'Send_data 1
     ClientWebSocket'Filter_messages 1
     [...]

  Live Transition Instances
     None
	\end{lstlisting}
	
	A dead marking is a marking from where no other markings can be reached.
	 In other words, there are no transitions
	for which there are enabled bindings, and the system is effectivley stopped.
	For our example, we have a single dead marking, and it is the same as our home
	marking, confirming that this is a leaf node in the tree.
	
	We also get a listing of dead transition instances, which are transitions that
	never have any enabled bindings in a reachable marking and are thus never
	fired. This can be useful to detect problems with a model, but in this example
	it is expected for many of the transitions, since we are not sending any kind
	of messages in the configuration considered.
	
	Last, there are live transition instances. A transition is live if from any
	reachable marking we can find an occurrence sequence containing the transition.
	Our example has no such transition, which follows trivially from the fact that
	there is a dead marking.

	The state space report also contains fairness properties, but this does
	not apply to our model sincethe state space does not contains cycles. We
	will not go into detail about this, and instead refer to \cite{cpn_book},
	chapter 7 for more information.
	
\subsection{Larger Configurations} 
	Above we have considered the simplest possible configuration of the WebSocket
	protocol model.Below we present the results from considering more complex
	configurations. For each configuration we only present select elements from the
	state space report

	\subsubsection{Configuration 1: One short message}
	The next step is to gradually increase the number of messages to be passed
	between the endpoints. We start by configuring the client to send a single
	message: |{Op=TEXT, Message="Short message"}|. The results are shown in
	\lstref{ssa_one_msg}
	
	\begin{lstlisting}[language={},float,label=lst:ssa_one_msg,caption=One message]
  State Space
     Nodes:  29
     Arcs:   28
     Secs:   0
     Status: Full

  Home Markings
     [29]

  Dead Markings
     [29]

  Live Transition Instances
     None

     No infinite occurrence sequences.
	\end{lstlisting}
	
	The number of markings has not increased by much, and the other properties are
	largely the same, except there are fewer dead transition instances. 
	The visualisation (\figref{ss_vis_1message}) shows there is still only one
	chain of occurrences, and by manual inspection of state 29 we verified that
	this state had the desired outcome without side effects (connection open and
	message received).
	\fig[0.4]{ss/vis_1Message.eps}{One message}{ss_vis_1message}
	
	\subsubsection{Configuration 2: One ping, then one message}
	When we add another message to be sent (a ping, which will also result in a
	pong being sent back), we see in \lstref{ssa_ping_msg} that the number of nodes
	has increased by an order of magnitude. This is due to the fact that for each
	position the first message can have, the other message can be
	anywhere from not sent yet to at the same place, yielding an exponential
	increase in possible states.
	
	\begin{lstlisting}[language={},float,label=lst:ssa_ping_msg,caption=One ping
	then one message] State Space
     Nodes:  475
     Arcs:   1140
     Secs:   1
     Status: Full
     
  Home Markings
     [475]

  Dead Markings
     [475]
	\end{lstlisting}
	
	CPN Tools supports exporting state spaces to a format supported by Graphviz, an
	open source application for visualising graphs. We have used Graphviz to
	visualise this state space, shown in \figref{ss_vis_1ping1message}. This
	clearly demonstrates the effect of the state space explosion problem. 
		The home marking 475 was again manually verified as correct.
	
	\fig[0.145]{ss/vis_1ping1message.eps}{State space for one ping, one
	message configuration}{ss_vis_1ping1message}

	\subsubsection{Configuration 3: One message, then one ping}
	By reversing the order of the two messages, the state space gets slightly
	larger, due to the fact that WebSocket could send the ping frame first, since
	it is a control frame. Results are shown in \lstref{ssa_msg_ping}.
	
	\begin{lstlisting}[language={},float,label=lst:ssa_msg_ping,caption=One
	message then one ping]
  State Space
     Nodes:  513
     Arcs:   1141
     Secs:   1
     Status: Full
     
  Home Markings
     None

  Dead Markings
     [512,513]
	\end{lstlisting}
	
	Note that we now no longer have any home marking, and instead have two dead
	markings, representing the two possible orderings of the two messages. In this
	situation it is possible to use the |HomeSpace| query to see if the markings
	belong to a so-called home space, meaning for any reachable marking in the
	state space, it is possible to reach at least one of the markings in the home
	space. The full query used is |HomeSpace(ListDeadMarkings())|, and it returned
	|true| when executed in this instance.
	
	\subsubsection{Configuration 4: One long message}
	We now set a message to be sent that is large enough to require fragmenting:
	
	|{Op=TEXT,Message="Very long message. Very long message. |
	
	|Very long message. Very long message. Very long message. "}|
	
	\begin{lstlisting}[language={},float,label=lst:ssa_long_msg,caption=One long
	message]
  State Space
     Nodes:  813
     Arcs:   2331
     Secs:   2
     Status: Full
     
  Home Markings
     [813]

  Dead Markings
     [813]
	\end{lstlisting}
	We see in \lstref{ssa_long_msg} that this has more states than sending two
	simple messages.
	We can also use the Best Upper Multi-set Bounds section, shown in
	\lstref{ssa_long_msg_frag} to find the largest collection of fragments. 
	
	\begin{lstlisting}[language={},gobble=1,float,label=lst:ssa_long_msg_frag,caption=Upper
	Multi-set Bounds - long message fragments]
	
	1`[WsFrame({Fin=clear,Rsv1=clear,Rsv2=clear,Rsv3=clear,Opcode=1,Masked=set,Payload_length=20,Masking_key=Mask([0,0,0,0]),Payload="Very long message. V"}),
	WsFrame({Fin=clear,Rsv1=clear,Rsv2=clear,Rsv3=clear,Opcode=0,Masked=set,Payload_length=20,Masking_key=Mask([0,0,0,0]),Payload="ery long message. Ve"}),
	WsFrame({Fin=clear,Rsv1=clear,Rsv2=clear,Rsv3=clear,Opcode=0,Masked=set,Payload_length=20,Masking_key=Mask([0,0,0,0]),Payload="ry long message. Ver"}),
	WsFrame({Fin=clear,Rsv1=clear,Rsv2=clear,Rsv3=clear,Opcode=0,Masked=set,Payload_length=20,Masking_key=Mask([0,0,0,0]),Payload="y long message. Very"}),
	WsFrame({Fin=set,Rsv1=clear,Rsv2=clear,Rsv3=clear,Opcode=0,Masked=set,Payload_length=15,Masking_key=Mask([0,0,0,0]),Payload=" long message. "})]
	\end{lstlisting}
	
	We are also able to inspect every other possible combination, and can thus
	confirm that messages are being split correctly.

	\subsubsection{Configuration 5: Ping, text, close}
	The client is now set to send three messages of different types: A ping
	(which will solicit a pong), a short text string, and a	close message. With
	three messages, state space calculation becomes noticeably time-consuming.
	\lstref{ssa_ping_text_close} shows the results.
	
	\begin{lstlisting}[language={},float,label=lst:ssa_ping_text_close,caption=Ping
	Text Close]
  State Space
     Nodes:  6129
     Arcs:   19625
     Secs:   14
     Status: Full
     
  Dead Markings
     6 [6129,6112,6029,5960,5632,...]
	\end{lstlisting}
	
	Here we have six dead markings. In three of the cases, the close is sent before
	the text. For each of those three, the three cases consist of the pong frame
	either successfully arriving at the client, not being received by the client due
	to the connection being closed, or not being sent from the server for the same
	reason. This was verified by manual inspection in CPN Tools. We also verified
	that this set is a home space.

	\subsubsection{Even larger configurations}
	We tried adding one more message to the server application, but after running for
	5 hours the state space calculation had still not been able to compute a
	complete state space. Fortunately, it is still possible to create a report for
	the partial graph, which we show in \lstref{ssa_large}. 
	
	\begin{lstlisting}[language={},float,label=lst:ssa_large,caption=Large
	configuration]
  State Space
     Nodes:  165748
     Arcs:   707380
     Secs:   18000
     Status: Partial
	\end{lstlisting}
	The report also showed that it had not detected any cycles, which reinforces
	the claim that the model works as it should.
	
	\subsection{Summary}
	State space analysis has proved to be very useful by uncovering problems
	with the WebSocket CPN model, and effectively allowing us to conclude that the
	model is now valid.  All but two of the transitions are enabled at some point;
	the two exceptions are \nodename{Fail} in the \nodename{ClientApplication}
	module and \nodename{Send Reject} in the \nodename{ConnectionResponse} module,
	which correctly are never enabled.
	This means we have full coverage of the model; no part of it is unused and
	unaccounted for.
