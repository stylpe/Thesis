\chapter{State Space Analysis of the CPN Web Socket Protocol}
\label{chap:statespace}

One of the advantages of Coloured Petri Nets is the ability to conduct state
space analysis, which can be used to obtain information about a model's
behavioural properties, and can be used to locate errors and increase
confidence in the correctness of the CPN model.

\section{State Spaces}
A state space is a directed graph where each node represents a reachable marking
(a state) and each arc represents an occurring binding element (a transition
firing with specific values bound to the variables of the transition). CPN Tools
by default generates the state space in breadth-first order. 

Once generated, the state space can be visualised directly in CPN Tools.
Starting with the node for the initial state, one can pick a node and show all
nodes that are reachable from it, and in this way explore the state space
visually. This can be very tedious and unmanageable for complex state spaces,
though, and instead it is usually better to use queries to automate the analysis
based on state spaces.

	\subsection{Strongly Connected Component graph}
	In graph theory, a strongly connected component (SCC) of a graph is a maximal
	subgraph where all nodes are reachable from each other. An SCC graph has a node
	for each SCC of the graph, connected by arcs determined by the arcs in the
	underlying graph between nodes that belong to different SCCs. An
	SCC graph is acyclic, and an SCC is said to be trivial if it contains only one
	node.
	
	By calculating the SCC graph of the state space, some of the further
	analysis becomes simpler and faster, such as determining reachability and
	cyclic behaviour.

	\subsection{Application of State Spaces}
		The biggest drawback of state space analysis is the size a state spaces may
		become very large. The number of nodes and arcs often grows exponentially with
		the number of starting system parameters. \com{Size of the midel
		configuration considered?}
		
		This can be remedied by picking smaller configurations,  that encapsulate
		different parts of the system. This was necessary with the WebSocket Protocol
		model, as the state space took too long to generate initially.
		
		A variant of this is situations where tokens can be generated an
		unlimited amount of times on a place, making the state space infinite. This
		can be remedied by modifying the model to limit the number of simultaneous
		tokens in the offending place.
		
		\com{Diskutere mer hva som måtte gjøres med WebSocket, illustrasjon}
		A model that incorporates random values is not fit for computing a state
		space, as the number of possible enabled bindings are arbitrary for a given
		state depending on the possible random values. Some values might not get used. Other
		times there is an infinite number of possible values (like a floating point
		number).
		
		\com{utpeke forskjell på non-deterministic og random}
		This can sometimes be alleviated by changing the behaviour to be
		deterministic, for example by replacing the random function with a small color
		set, such as an index or an integer with bounds. This lets the state space
		generator create bindings for every possible value. 
		
		For the WebSocket Protocol model, this was a problem for the masking key in
		WebSocket frames, which is supposed to be a random 4-byte string. The
		randomisation function was simply changed to always return four zeros.

\section{State space report}
Once the state space has been generated, CPN Tools lets the user save a
state space report as a text document. The report is split into parts that each
describe different aspects about the state space.

To explain each section of the state space report, a simple report for the
WebSocket protocol has been generated, in a configuration where no messages are
set to be sent. Thus, the only thing that will happen is that a connection will
be established. Later in the chapter we will consider more elaborate
configurations of the WebSocket protocol.
	
	\subsection{Statistics}
	The first section describes general statistics about the state space.
	\begin{lstlisting}
  State Space
     Nodes:  17
     Arcs:   16
     Secs:   0
     Status: Full

  Scc Graph
     Nodes:  17
     Arcs:   16
     Secs:   0

	\end{lstlisting}
	This state space has 17 possible markings, with 16 enabled transition
	occurrences connecting them. There is one more node than there are arcs, which
	means this graph is a tree.
	
	The |Secs| field shows that it took less than one second to calculate this
	state space, while the |Status| field tells whether the report is generated
	from a partial or full state space.
	
	We also see that the Scc Graph has the same number of nodes and arcs, meaning
	that there are no cycles in the state space (although this was already known
	from the fact that it is a tree).
	
	\subsection{Boundedness Properties}
	This section describes the minimum and maximum number of tokens for
	each place in the model, as well as the actual tokens these places can have.
	The text has been reformatted and truncated for readability.
	\begin{lstlisting}[language={}]
  Best Integer Bounds
                             Upper      Lower
     ClientApplication
          Active_Connection 1
                             1          0
          Conn_Result 1
                             1          0
          Connection_failed 1
                             0          0
          Messages_received 1
                             1          1
          Messages_to_be_sent 1
                             0          0
     .....

  Best Upper Multi-set Bounds
     ClientApplication
          Active_Connection 1
                         1`()
          Conn_Result 1
                         1`success
          Connection_failed 1
                         empty
          Messages_received 1
                         1`[]
          Messages_to_be_sent 1
                         empty
     .....
     ClientWebSocket
          Connection_status 1
                         1`CONN_OPEN
     .....
     ServerWebSocket
          Connection_Status 1
                         1`CONN_OPEN
     .....

  Best Lower Multi-set Bounds
     ClientApplication
          Active_Connection 1
                         empty
          Conn_Result 1
                         empty
          Connection_failed 1
                         empty
          Messages_received 1
                         1`[]
          Messages_to_be_sent 1
                         empty
     .....
     ClientWebSocket
          Connection_status 1
                         empty
     .....
     ServerWebSocket
          Connection_Status 1
                         empty
     .....
	\end{lstlisting}
	
	Many places show a lower and upper bound of 1. This shows a weakness
	in the approach of using lists to facilitate ordered processing of tokens: We
	can't see the actual number of tokens that are in the place, because
	technically there is just a list there. 
	
	Apart from that, we see that both the client and the server has an open
	connection at some point, as the |Connection_status| place on the
	|ClientWebSocket| and |ServerWebSocket| pages have both had a |CONN_OPEN|
	token.
	
	\subsection{Home Properties}
	This section shows all home markings. A home marking is a marking that can
	always be reached no matter where we are in the state space. \com{Vise marking}
	\begin{lstlisting}[language={}]
  Home Markings
     [17]
	\end{lstlisting}
	We see that there is one such marking at node 17. From earlier we know that the
	state space is a tree, and if this node is always reachable it must be a leaf
	and all the other nodes must be in a chain. This tells us that there is only
	one possible sequence of transitions to establish a connection. We can then
	confidently say that the model works correctly with this configuration.
	
	\subsection{Liveness Properties}
	This section describes liveliness of the state space. Some of the transitions
	have been omitted for readability.
	
	\begin{lstlisting}[language={}]
  Dead Markings
     [17]


  Dead Transition Instances
     ClientApplication'Fail 1
     ClientApplication'Receive_data 1
     ClientApplication'Send_data 1
     ClientWebSocket'Filter_messages 1
     .....

  Live Transition Instances
     None
	\end{lstlisting}
	
	A dead marking is a marking from where no other markings can be reached.
	\com{from where? for which? formuler} In other words, there are no transitions
	for which there are enabled bindings, and the system is effectivley stopped.
	For our example, we have a single dead marking, and it is the same as our home
	marking, confirming that this is a leaf node in the tree.
	
	We also get a listing of dead transition instances, which are transitions that
	never have any enabled bindings and are thus never fired. This can be useful to
	detect problems with a model, but in this example it is expected for many of
	the transitions, since we are not sending any kind of messages in this configuration.
	
	Last, there are live transition instances. A transition is live if we from any
	reachable marking can find an occurrence sequence containing the transition.
	Our example has no such transition, which follows trivially from the fact that
	there is a dead marking.

	The State Space report also contains fairness properties, but this does
	not apply to our model since it contains no loops. We will not explain more
	about this here, and instead refer you to \cite{cpn_book} chapter 7 for more
	information.
	
\section{TODO:} 
Skrive om resten av analysene.

Flette inn dette:

\subsection{Error discovery}
An error in the model was discovered this way, in the Unwrap and Receive module,
where the pong reply was adding a new list instead of appending to the old one
in outgoing messages. \com{Flere detaljer - hvordan viste fielen seg? Figur,
før og etter}
