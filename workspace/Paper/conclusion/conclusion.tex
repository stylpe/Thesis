\chapter{Conclusion}
\label{chap:conclusion}


\section{Results}

We have constructed a fully functional proof of concept that demonstrates the
advantages and disadvantages of using ontologies to manage code generation
pragmatics and applying them to a cpn model. Through our evaluation by using it to annotate CPN models
of different sizes, we have shown that ontology reasoning using CWA yields
decreasing performance to a point that makes our prototype unusable for the
WebSocket CPN model.

At the time of writing, we are using unreleased development versions of parts of
ePNK, and are therefore unable to make \thename{} avaialble for download as an
Eclipse plugin.
ePNK is scheduled for an update release later this summer.

\subsection{Limitations}

The graphical diagram editor of ePNK has some shortcomings and we do not
conisder it fully mature. Arc appearance is not serialised, making it impossible
to accurately copy layout from CPN Tools. Labels are not attached to their
parent elements, and by default are placed at coordinates 0,0. We look forward
to improvements to this part of ePNK.

Accurate modeling of module instancing (the fact that a module can be a
submodule of several modules) is impractical in the current version of ePNK if
we are to only subclass existing leaf classes.
RefPlaces can only reference a single other place. It could be used to represent
either the port place or the socket place, but either case presents problems. As
a port place, it could be on a module that has several instances, so it would
either need to contain a reference for each instance (which is not possible due
to being limited to 1) or the module itself would need to be duplicated (which
is impractical from a user perspective since all annotations would have to be
placed once on each module). As a socket place, it could be connected to several
submodules (as is the case with the Connection status place in the WebSocket
CPN model) and would need to have a reference to the ports in each of them. 

\section{Future work}

Continue the work of Simonsen, defining more domain-specific pragmatics

Expand generic pragmatics to describe pragmatics properties

Developing ontology reasoner tailored for CWA.

Integrating a better ontology editor tailored for pragmatics

Implement structured labels for the CPN Type.

When importing, let user choose between CPN and Pragma CPN. Alternatively,
mechanic for upgrading the Type.

\section{Acknowledgments}

Thank supervisor Lars Michael Kristensen

Kent Inge Fagerland Simonsen

Michael Westergaard for help with Access/CPN

Ekkart Kindler for extensive help with ePNK