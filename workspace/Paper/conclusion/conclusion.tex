\chapter{Conclusion}
\label{chap:conclusion}

With \thename{}, we have constructed a fully functional proof of concept that
demonstrates the advantages and disadvantages of using ontologies to manage code
generation pragmatics and applying them to a CPN model. As shown in Chapter
\ref{chap:analysis}, this prototype meets all of the requirements in Section
\ref{sec:requirements}. Through our evaluation by using it to annotate CPN
models of different sizes, we have shown that using ontologies gives a highly
expressive means to defining pragmatics as well as providing automatic
validation of the annotated model. However ontology reasoning using CWA yields
decreasing performance as model size increases, to a point that makes our
prototype unusable for the WebSocket CPN model. Still, we believe our approach
has potential, and warrants further optimisation and research.

\section{Future work}

Continue the work of Simonsen, defining more general and domain-specific
pragmatics. Also, assert required configuration arguments.

Finding or developing ontology reasoner tailored for CWA.

A specialised tool for creating model specific pragmatics would have been ideal.
Such a tool would give simple mechanics using GUI controls for specifying which
model elements a pragmatic can be attached to, and which parameters it has. The
Plugin Manifest editor is a good example of what we have in mind. However, this
could not be included for this thesis due to time constraints.

Implement structured labels for the CPN Type. Could enable placement of
pragmatics inside expressions.

When importing, let user choose between CPN and Pragma CPN. Alternatively,
mechanic for upgrading the Type.

Accurately model sub-modules, ports and sockets. This enables more advanced
pragmatics; for instance the Principal pragmatic could be specified to only be
available for substitution transitions on top-level modules.

\section{Acknowledgments}

Thank supervisor Lars Michael Kristensen

Kent Inge Fagerland Simonsen 

Michael Westergaard for help with Access/CPN

Ekkart Kindler for extensive help with ePNK