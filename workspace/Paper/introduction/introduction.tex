\chapter{Introduction}
\label{chap:introduction}
 
Software engineering is an increasingly complex discipline, with
new and improved technology emerging at a rapid pace. There is no single answer
on how to approach all problems, which has lead to the development of several
software development paradigms. A motivation common to many of them is that
software developers have always sought increasing levels of abstraction.
Today's technology is at a level that potentially gives us means for
automatically generating source code from conceptual domain models of
applications, and substantial research is being conducted to create formal
methods for unleashing this potential.


\section{Model Driven Software Engineering}

Models and diagrams have been used in software design for a long time, and have
been standardised with the introduction of Unified Modeling Language (UML)
and the methods and tools developed around it (like Unified Process).
However, models largely play a secondary role, performing as design tools and
documentation.

Model Driven Engineering (MDE) has emerged as a new development methodology,
putting the models in the center of the software development process. Developers
design models that serve both as documentation and as a basis for
implementation, and become a layer of abstraction over source code. This
trend has been touted as a new programming paradigm, the same way
object-oriented programming was when it was conceived.

MDE has two central concepts:
\begin{itemize}
	\item Domain specific modeling languages (DSML), which are used to formalise application structure, behavior,
and requirements of specific domains, such as financial services, warehouse
management, task scheduling, and protocol and communication software. A DSML
relies on a metamodel to describe concepts of the domain, along with associated
sematics and constraints.
	\item  Transformation engines and generators, that process models to produce various
artifacts, including written documentation, deployment descriptions,
alternate representations, and source code.\com{Mer}
\end{itemize}

Meta-modeling, creating a DSML. Model for OO (figure). \com{TODO}

Using models as the core design element comes with several benefits. It allows
developers to employ different methods of analysis of the models, like verifying
correctness, completeness, finding race conditions, and analysing scalability.
Models also act as graphical representations of the system, making them more understandable
for people that are not programmers. 

One of the central arguments for MDE is automatic source code generation.
Several advantages come from this: Documentation and implementation are 
synchronised, boilerplate code and automatic testing is taken care of\ldots
\com{Mer}

One modeling language that is being used as a DSML is Petri nets. Which kind of
systems, concurrent.
Enables verification and analysis, State Spaces. \com{Mer}

There exist many extensions to the Petri Net formalism, termed high-level Petri
Nets, that define additional constructs or that change or enhace concepts of
Petri Nets. One of these extensions is called Colored Petri Nets (CPN). The term
Colored comes from the fact that a token can have a color from a defined color
set, essentially data values from a set of values. A common way of analysing
Petri Nets is called state space exploration, and is a powerful mothod for
automatic model verification and determination of several properties. This
thesis gives a short introduction to CPN and state space exploration.

\section{Related Work}

A CPN model can accurately model many types of software systems, but cannot
directly be used to generate a software implementation. Research is being
conducted to develop approaches for and demonstrate how to use CPN and other
Petri Net variants to model software and generate source code. 

Kristensen and Westergaard \cite{kristensen2010automatic} examine challenges of
using CPN for automatic code generation, and propose a new Petri Net type called
Process-Partitioned CPNs. They demonstrate and evaluate it by
designing an implementation for the Dynamic MANET On-demand (DYMO) routing
protocol.

Mortensen \cite{mortensen2000automatic} presented an extension to the Design/CPN
tool to support automatic implementation of systems by reusing the model
simulation algorithm, thus eliminating the usual manual implementation phase.
They demonstrate the tool by implementing an access control system, and
evaluate benefits of the model architecture.

Lassen and Tjell \cite{lassen2010automatic}  present a method for developing
Java applications from Colored Control Flow Nets (CCFN), a specialised type of
CPN. CCFN forces the modeler to describe the system in an
imperative manner, making it easier to automatically map to Java code.

This thesis focuses on work done by Simonsen \cite{Simonsen2011}, who discusses
some of the challenges in modelling and automatically generating software in the
domain of communication protocols, using the Kao-Chow authentication protocol
as an example.
The ideas introduced in the paper outline a method for annotating CPNs with a
set of code generation pragmatics that describe how model elements relate to and
bind to source code. 

Simonsen's approach consists of three parts: 
\begin{itemize}
	\item Annotate the CPN protocol model with pragmatics which bind the model
	entities to program concepts,
	\item Create a platform	model that knows how to implement specific constructs
	for a particular platform,
	\item Create a configuration model for deciding implementation details,
	depending on the protocol model and platform model.   
\end{itemize}   

This research defines the domain of the work for this thesis.

\section{Thesis Aims and Results}
The work of Simonsen is focused on the model transformation and code generation
aspect. Annotation of the CPN models are done with simple text files. This
approach is cumbersome, and there is a need for developing specialised
tool support for this purpose.

This thesis seeks to satisfy this need by answering the following
question: What is an approach for annotating Coloured Petri Nets with code
generation pragmatics following the idea of?

The method used to answer this question is to create a prototype application
framework, and evaluate it by annotating an example CPN model of the
WebSocket Protocol.

The code generation pragmatics (or just ``pragmatics'') are categorised
into three classes.
\begin{description}
	\item[General pragmatics] are used to define protocol entities, communication
	channels external method calls and API entry points for operations like opening
	or closing a connection, and sending or receiving data.
	\item[Domain specific pragmatics] are pragmatics that apply to all (or many)
	protocols within a particular domain. An example is security protocols, where
	examples of domain specific pragmatics relate to operations such as
	encryption, decryption, and nonce generation.
	\item[Model specific pragmatics] apply only to the specific model instances in
	which they are defined, and are used to label concepts unique to that model. 
\end{description}

The prototype builds on the ePNK framework,
which uses the Eclipse Modeling Framework (EMF) to provide an extensible
platform for working with CPN models. The prototype is desgned as a plugin for
Eclipse, and can import models created by CPN Tools. It lets the user annotate
the models with pragmatics through a tree editor. Pragmatics are defined using
ontologies, and can be dynamically loaded into models. Evaluation is done by
applying it to a set of protocol CPN models.

\section{Thesis Organisation}

The thesis is organised as follows:

\begin{description}
\item[Chapter~\ref{chap:background}:~\nameref{chap:background}] Provides a
description of the WebSocket protocol, the primary case study used in this thesis. An
introduction to Colored Petri Nets and the CPN Tools application used to
create the models, with a breakdown of the CPN model of the WebSocket protocol
we produced for the thesis.
\item[Chapter~\ref{chap:statespace}:~\nameref{chap:statespace}] Gives an
introduction to State Space Analysis of CPN models, and an example of how to
apply it using the model of the WebSocket Protocol as an example. This
establishes the produced model as correct. \com{Utvid litt, formuler bedre}
\item [Chapter~\ref{chap:technology}:~\nameref{chap:technology}] The produced
prototype is built on top of a number of software frameworks and technologies.
This chapter gives an introduction to these as well as the reasons for choosing them.
The Eclipse Platform and its modules; the Eclipse Modeling Framework; the ePNK
framework, which makes up the foundation of the prototype, and how it can be
extended to support new functionality; Access/CPN, the engine used
to omport models from CPN Tools; and an introduction to ontologies, the format
used to specify pragmatics classes.
\item [Chapter~\ref{chap:analysis}:~\nameref{chap:analysis}] Start with
discussion and detailing of requirements for the editor. 
We give details of the ePNK Petri Net Type Definition for Colored Petri Nets and
Annotated Colored Petri Nets. Describe how the implementation works.
Eclipse Plugin structure. Extension Points, used to register with ePNK
and extend context menus. Reasoning with ontologies, using this to decide
available pragmatics.
\item [Chapter~\ref{chap:evaluation}:~\nameref{chap:evaluation}] Discuccion on
which requirements have been met. Overview and explanation of test cases.
Results from test cases and feedback from users. Evaluation of technology used,
and opinion on maturity of MDD and the tools available.
\item [Chapter~\ref{chap:conclusion}:~\nameref{chap:conclusion}] Summary,
personal experience, limitations and suggested focus of future work.
\end{description} 


The reader is assumed to be familiar with Java programming, and the TCP/IP
Protocol Suite. Some basic knowledge of Petri Nets is also an advantage, but not
a strict requirement as we briefly introduce the basic constructs of the CPN
modeling language.
