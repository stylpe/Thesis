\chapter{Introduction}
\label{chap:introduction}
 
Software engineering is an increasingly complex discipline, with
new and improved technology emerging at a rapid pace. There is no single answer
on how to approach every challenge in modern software engineering, which has
lead to the development of several software development paradigms. A motivation common to many of them is that
software developers have always sought increasing levels of abstraction.
Today's software development technology is at a level that potentially gives us
means for automatically generating source code from conceptual domain models of
applications, and substantial research is being conducted to create formal
methods for unleashing this potential.


\section{Model Driven Software Engineering}

Models and diagrams have been used in software design for a long time, and have
been standardised with the introduction of Unified Modeling Language (UML)
\cite{umlInfra} and the methods and tools developed for and around it (like
Rational Unified Process \cite{rup}).
However, models largely play a secondary role, primarily playing the role of
design tools and documentation.

Model Driven Engineering (MDE) \cite{kent2002model} has emerged as a new development
methodology, putting the models in the center of the software development process. Developers
design models that serve both as documentation and as a basis for
implementation, and provides a layer of abstraction over source code. This
trend has been touted as a new programming paradigm, in the same way
object-oriented programming was when it was conceived.

MDE is based on two central concepts:
\begin{itemize}
	\item Domain specific modeling languages (DSML), which are used to formalise
	application structure, behavior, and requirements of specific domains, such as
	financial services, warehouse management, task scheduling, and protocol and
	communication software. A DSML relies on a metamodel to describe concepts of
	the domain, along with associated sematics and constraints.
	\item Transformation engines and generators, that process models to produce
	various artifacts, including documentation, deployment descriptions,
	alternative representations, and source code.\com{Mer}
\end{itemize}

Meta-modeling, creating a DSML. Model for OO (figure). \com{TODO}

Using models as the core design element comes with several benefits. It allows
developers to employ different methods of analysis of the models, like verifying
correctness, completeness, finding race conditions, and analysing scalability.
Models may also act as graphical representations of the system, making them more
understandable for people that are not programmers. This is for instance
important when soliciting requirements from customers.

One of the central arguments for MDE is automatic source code generation.
Several advantages come from this: Documentation and implementation are 
synchronised, boilerplate code and automatic testing is taken care of\ldots
\com{Mer}

One modeling language that is being used as a DSML is Petri Nets. 
Petri Nets are a type of directed graph used to model processes,
especially with an asynchronous and/or concurrent nature. Common
example domains are communication networks and protocols, as
well as concurrent programming design.
Petri Nets are very well suited for MDE, as they have robust methods for
computer-aided verification and analysis. 

There exist many extensions to the Petri Net formalism that define additional
constructs or that change or enhace concepts of Petri Nets. One of these
extensions is called Coloured Petri Nets (CPN). The term Coloured comes from
the fact that a token can have a colour from a defined colour set (type),
essentially data values from a set of values. Combined with the capability to
model synchronicity, concurrency and communication that Petri Nets give, the
functional programming language Standard ML \cite{milner1997definition} is used as a foundation
to define colours and colour sets in a compact manner, and to provide basic data
types and manipulation implementation.

A common way of analysing
Petri Nets is called state space exploration, and is a powerful method for
automatic model verification and determination of several properties. This
thesis gives a short introduction to CPN and state space exploration.

%\subsection{CPN Tools}

CPN Tools \cite{cpntools} is a popular graphical
tool for working with CPN models, from construction and simulation, to analysis
via state space exploration. CPN Tools uses the CPN ML language to specify
declarations and net inscriptions.
This language is an extension of Standard ML, developed at Edinburgh
University. \com{Mer om CPN Tools}
	
A CPN model can accurately model many types of software systems, but cannot
directly be used to generate a software implementation. Research is being
conducted to develop approaches for and demonstrate how to use CPN and other
Petri Net variants to model software and automatically generate source code.


\section{Thesis Aims and Results}
This thesis focuses on work done by Simonsen \cite{Simonsen2011}, who discusses
some of the challenges in modelling and automatically generating software. His
work is focused on the domain of communication protocols, and uses the Kao-Chow
authentication protocol as an example.
The ideas introduced in the paper sketch a method for annotating CPNs with a
set of code generation pragmatics that describe how model elements relate to and
bind to source code. Simonsen's approach consists of three parts: 
\begin{itemize}
	\item Annotate the CPN protocol model with pragmatics which bind the model
	entities to program concepts,
	\item Create a platform	model that describes how to implement specific
	constructs for a particular platform,
	\item Create a configuration model for capturing implementation details,
	including the mapping from the protocol model to the platform model.   
\end{itemize}  

The work of Simonsen \cite{Simonsen2011} is focused on the model transformation
and code generation aspect. Annotation of the CPN models in \cite{Simonsen2011}
are done with simple text files. This approach is cumbersome, and there is a
need for developing specialised tool support for this purpose.

Based on the very initial ideas of \cite{Simonsen2011}, the aim of this thesis
is to investigate how to support annotation of CPN models with code generation
pragmatics in a flexible and model-centric manner. By model-centric we mean that
annotation of CPN models should be closely integrated with editing of CPN model
elements.

The research method used to answer this question is to create a prototype
application framework, and evaluate it by annotating a selected set of CPN
models of communication protocols, including a detailed case study on the
application of the WebSocket protocol. The resulting application has been named
\thename{}. This name was chosen to resemble the naming format of other tools that
are built around CPN Tools, including Access/CPN and Design/CPN.

The requirements for the prototype can be divided into four main items:

\com{Mer detaljer}
\begin{itemize} 
	\item Importing models created in CPN Tools. 

	\item Annotate model with pragmatics. 
		
	\item Load sets of domain-specific pragmatics to make available for the model.
	
	\item Define set of model-specific pragmatics while annotating the model

\end{itemize}

The code generation pragmatics (or just ``pragmatics'') in the approach
developed in this thesis are categorised into three classes.
\begin{description}
	\item[General pragmatics] are used to define protocol entities, communication
	channels external method calls and API entry points for operations like opening
	or closing a connection, and sending or receiving data.
	\item[Domain specific pragmatics] are pragmatics that apply to all (or many)
	protocols within a particular domain. An example is security protocols, where
	examples of domain specific pragmatics relate to operations such as
	encryption, decryption, and nonce generation.
	\item[Model specific pragmatics] apply only to the specific model instances in
	which they are defined, and are used to label concepts unique to that model. 
\end{description}

\thename{} builds on the ePNK framework \cite{kindler2011epnk}, which uses the
Eclipse Modeling Framework (EMF) to provide an extensible platform for working with CPN models. The
prototype is desgned as a plugin for the Eclipse platform, and can import models
created by CPN Tools. It lets the user annotate the models with pragmatics through a tree
editor. Pragmatics are defined using ontologies, which gives advanced
capabilites to classify pragmatics and dynamically deduce appropriate
pragmatics for individual model elements. These ontologies can be dynamically
loaded into models, which lets the user write their own ontology containing
model specific pragmatics.

\section{Related Work}

Kristensen and Westergaard \cite{kristensen2010automatic} examine challenges of
using CPN for automatic code generation, and propose a new Petri Net type called
Process-Partitioned CPNs. They demonstrate and evaluate it by
designing an implementation for the Dynamic MANET On-demand (DYMO) routing
protocol.

Mortensen \cite{mortensen2000automatic} presented an extension to the Design/CPN
tool to support automatic implementation of systems by reusing the model
simulation algorithm, thus eliminating the usual manual implementation phase.
They demonstrate the tool by implementing an access control system, and
evaluate benefits of the model architecture.

Lassen and Tjell \cite{lassen2010automatic}  present a method for developing
Java applications from Coloured Control Flow Nets (CCFN), a specialised type of
CPN. CCFN forces the modeler to describe the system in an
imperative manner, making it easier to automatically map to Java code.

\section{Thesis Organisation}
The thesis is organised as follows:

\begin{description}
\item[Chapter~\ref{chap:background}:~\nameref{chap:background}.] Provides a
description of the WebSocket protocol, the primary case study used in this
thesis. The chapter gives an introduction to Coloured Petri Nets (CPNs) and the
CPN Tools application used to create the models, combined with a detailed
presentation of the CPN model of the WebSocket protocol produced as part of this
thesis.
\item[Chapter~\ref{chap:statespace}:~\nameref{chap:statespace}.] Gives an
introduction to state space analysis of CPN models, and an example of how to
apply it using the model of the WebSocket Protocol. This validates that the
constructed CPN model of the WebSocket protocol is behaviourally correct.
\item [Chapter~\ref{chap:technology}:~\nameref{chap:technology}.] \thename{} is built
on top of a number of software frameworks and technologies.
This chapter gives an introduction to these as well as the reasons for choosing
them: The Eclipse Platform and its modules; the Eclipse Modeling Framework; the ePNK
framework (which makes up the foundation of \thename{}); Access/CPN (the engine
used to import models from CPN Tools). This chapter also provides an
introduction to ontologies, the format used to specify pragmatics classes.
\item [Chapter~\ref{chap:analysis}:~\nameref{chap:analysis}.] Gives a
discussion and detailing of requirements for \thename{}. 
We give details of the ePNK Petri Net Type Definition for Coloured Petri Nets and
Annotated Coloured Petri Nets. Describe how the implementation works.
Eclipse Plugin structure. Extension Points, used to register with ePNK
and extend context menus. Reasoning with ontologies, using this to decide
available pragmatics. \com{Bedre flyt her}
\item [Chapter~\ref{chap:evaluation}:~\nameref{chap:evaluation}.] Discuccion on
which requirements have been met. Overview and explanation of test cases.
Results from test cases and feedback from users. Evaluation of technology used,
and opinion on maturity of MDD and the tools available.
\item [Chapter~\ref{chap:conclusion}:~\nameref{chap:conclusion}.] Summary,
personal experience, limitations and suggested focus of future work.
\end{description} 


The reader is assumed to be familiar with Java programming, the basics of
functional programming languages, and the TCP/IP Protocol Suite. Some basic
knowledge of Petri Nets is also an advantage, but not a strict requirement as we
briefly introduce the basic constructs of the CPN modeling language as part of
describing the WebSocket protocol in Chapter~\ref{chap:background}.
