\chapter{Introduction}
\label{chap:introduction}
 
The art of software development is an increasingly complex discipline, with
new and improved technology emerging at a rapid pace. There is no single answer
to how to approach all problems, which has lead to the development of several
software development paradigms. But a motivation common to all of them is that
developers have always sought increasing levels of abstraction. Today's
technology is at a level that potentially gives us means of generating source
code from conceptual models of applications, and much research is being
conducted to create formal methods of applying this potential.


\section{Model Driven Software Engineering}

Models and diagrams have been used in software design for a long time, and have
been standardised with the introduction of Unified Modeling Language (UML)
\com{ref} and the tools developed around it (like Unified Process). However,
they largely play a secondary role, performing as design tools and
documentation.

Model Driven Engineering (MDE) has emerged as a new development methodology,
putting the models in the center of the development process. Developers design
models that act as both documentation and implementation, and become a layer of
abstraction over source code. This has been hailed as a new paradigm shift,
like the transition from procedural programming to object-oriented programming was.

MDE has two central concepts:
- Domain specific modeling languages (DSML), which are used to formalise application structure, behavior,
and requirements of specific domains, such as financial services, warehouse
management, task scheduling, and protocol and communication software. A DSML
uses a metamodel to describe concepts of the domain, along with associated
sematics and constraints. 

- Transformation engines and generators, that process models to produce various
artifacts, including written documentation, deployment descriptions,
alternate representations, and source code.

Meta-modeling, creating a DSML. Model for OO (figure).

Using models as the core design element comes with benefits. It allows
developers to employ different methods of analysis of the models, like verifying
correctness, completeness, finding race conditions, and measuring scalability. Models also
act as graphical representations of the system, making them more understandable
for people that are not programmers. 

One of the central arguments for MDE is automatic source code generation.
Several advantages come from this: Documentation and implementation are kept in
sync, boilerplate code and automatic testing is taken care of 

A modeling language, Petri nets. Which kind of systems, concurrent. 
Enables verification and analysis, State Spaces. 

There exist many extensions to the Petri Net formalism, termed high-level Petri
Nets, that define additional constructs or that change or enhace concepts of
Petri Nets. One of these extensions is called Colored Petri Nets (CPN). The term
Colored comes from the fact that a token can have a color from a defined color
set, essentially data values from a set of values. This thesis gives an
introduction to CPN and methods of analysing such models.


\section{Related work}
There are examples that demonstrate how to use CPN and other Petri Net variants
to model software and generate source code. Kristensen and Westergaard
\cite{kristensen2010automatic} examine challenges of using CPN for automatic
code generation, and propose a new Petri Net type called Process-Partitioned
CPNs. They demonstrate and evaluate it by implementiong the Dynamic MANET
On-demand (DYMO) routing protocol.

Mortensen \cite{mortensen2000automatic} presented an extension to the Design/CPN
tool to support automatic implementation of systems by reusing the model
simulation algorithm, thus eliminating the usual manual implementation phase.
They demonstrate the tool by implementing an access control system, and
evaluiate benefits of the model architecture.

Lassen and Tjell \cite{lassen2010automatic}  present a method for developing
Java applications from Colored Control Flow Nets (CCFN), a specialised type of
CPN. CCFN coerces the modeler to describe the system in an
imperative manner, making it easier to automatically map to Java code.

This thesis focuses on work done by  K. I. F. Simonsen \cite{Simonsen2011}. This
paper discusses some of the challenges in modelling and automatically
generating software in the domain of protocols, using the Kao-Chow
authentication protocol as an example. The paper outlines a method for
annotating CPNs with a set of code generation pragmatics that describe how model
elements relate to and  bind to source code. This research defines the domain
for the work of this thesis.

Simonsen's approach consists of three parts: 
\begin{itemize}
	\item Annotate the CPN protocol model with
pragmatics which bind the model entities to program concepts,
	\item create a platform
model that knows how to implement specific things for a particular platform,
	\item and
create a configuration model for deciding implementation details, depending on
the protocol model and platform model.   
\end{itemize}   

The code generation pragmatics (or just ``pragmatics'') are categorised
into three classes.
\begin{description}
	\item[General pragmatics] are used to define entities, communication channels
	external method calls and API entry points for operations like opening or
	closing a connection, and sending or receiving data.
	\item[Domain specific pragmatics] are pragmatics that apply to all (or many)
	protocols within a potential domain. An example is security protocols, where
	examples of domain specific praagmatics relate to operations such as
	encryption, decryption, and nonce generation.
	\item[Model specific pragmatics] apply only to the specific model instances in
	which they are defined, and are used to label concepts unique to that model. 
\end{description}

\section{Thesis Aims}
The work of Simonsen is focused on the model transformation and code generation
aspect. Annotation of the CPN models are done with simple text files. This
approach is cumbersome, and there is a need for a specialised tool for this
purpose.

This thesis seeks to satisfy this need by answering the following
question: What is a good method of annotating Coloured Petri Nets with code
generation pragmatics?

The method used to answer this question is to create a prototype application
framework, and evaluate it by using it to annotate an example model of the
WebSocket Protocol.

The framework should be capable of supporting the different classes of
pragmatics, categorised into General, Domain Specific, and Model Specific
Pragmatics.

It should be possible to import CPN models created in CPN Tools. Annotations are
placed through a tree editor, and also, if possible, through a graphical diagram
editor. The framework should provide contextual content assistance

\section{Thesis Organisation}

The thesis is organised as follows:
\com{6-7 linjer om hvert kapittel.}
\begin{description}
\item[Chapter~\ref{chap:background}:~\nameref{chap:background}] A description
of the WebSocket protocol, the primary case study used in this thesis. An
introduction to Colored Petri Nets and the CPN Tools application used to
create the models, with a breakdown of the CPN model of the WebSocket protocol
we produced for the thesis.
\item[Chapter~\ref{chap:statespace}:~\nameref{chap:statespace}] An explanation
of State Space Analysis of CPN models, and an example of how to apply it using
the WebSocket Protocol as the subject. This establishes the produced model as
correct.
\item [Chapter~\ref{chap:technology}:~\nameref{chap:technology}] The produced
prototype is built on top of a number of frameworks and technologies. This
chapter gives an introduction to each of them, including the Eclipse Platform
and its modules. We describe ePNK, which makes up the foundation of the
prototype, and how it can be extended to support new functionality. Explanation
of reason behind each chosen solution. Introduction to ontologies, the format
used to represent pragmatics classes.
\item [Chapter~\ref{chap:analysis}:~\nameref{chap:analysis}] Start with
discussion and detailing of requirements for the editor. 
We give details of the ePNK Petri Net Type Definition for Colored Petri Nets and
Annotated Colored Petri Nets. Describe how the implementation works.
Eclipse Plugin structure. Extension Points, used to register with ePNK
and extend context menus. Reasoning with ontologies, using this to decide
available pragmatics.
\item [Chapter~\ref{chap:evaluation}:~\nameref{chap:evaluation}] Discuccion on
which requirements have been met. Overview and explanation of test cases.
Results from test cases and feedback from users. Evaluation of technology used,
and opinion on maturity of MDD and the tools available.
\item [Chapter~\ref{chap:conclusion}:~\nameref{chap:conclusion}] Summary,
personal experience, limitations and suggested focus of future work.
\end{description} 


The reader is assumed to be familiar with Java programming. \com{mer?}