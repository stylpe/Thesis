\chapter{Introduction}
\label{chap:introduction}

Software development paradigms, 

\section{Model Driven Software Development}

Short explanation of MDD

\subsection{Code Generation}

Short intro to code generation from models, arguments for 

\section{Related work}

References to other works and theses, explaining their approaches

Reference and longer explanation of current work of Simonsen, which defines the
domain for the prototype

\section{Thesis Aim}

What is a good way of annotating Coloured Petri Nets with code generation
pragmatics?

This thesis seeks to answer this question by creating a prototype application
and using it to annotate a test model.

The framework should be capable of supporting several classes of pragmatics, 
categorised into General, Domain Specific, and Model Specific Pragmatics.

It should be possible to import CPN models created in CPN Tools. Annotations are
placed through a tree editor, and also, if possible, through a graphical diagram
editor. Errors highlighting?

Finally, there should be an easy way to invoke the code generator from the
editor.

\section{Thesis organisation}

The thesis is described below:

\begin{description}
\item[Chapter~\ref{chap:background}:~\nameref{chap:background}] Introduction to
CPN, showing how to use it by modeling the WebSocket protocol as an example.
Overview of related research. Describe CPN Tools.
\item[Chapter~\ref{chap:statespace}:~\nameref{chap:statespace}] An explanation
and application of state space analysis using the WebSocket Protocol as the
subject.
\item [Chapter~\ref{chap:technology}:~\nameref{chap:technology}] Explanation of
all technologies used to build then prototype, including the Eclipse Platform
and its modules, which make up the foundation of the prototype.
\item [Chapter~\ref{chap:analysis}:~\nameref{chap:analysis}] Discussion and
detailing of requirements for the editor. Overview and explanation of test cases.
\item [Chapter~\ref{chap:implementation}:~\nameref{chap:implementation}]
Describe how the implementation works. Explanation of chosen solutions to requirements. 
\item [Chapter~\ref{chap:evaluation}:~\nameref{chap:evaluation}] Discuccion on
which requirements have been met. Results from test cases and feedback from
users.
\item [Chapter~\ref{chap:conclusion}:~\nameref{chap:conclusion}] Summary,
personal experience, limitations and suggested focus of future work.
\end{description} 

\section{Required Knowledge}

The reader is assumed to be familiar with Java programming.