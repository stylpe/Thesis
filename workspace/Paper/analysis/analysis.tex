\chapter{Analysis and Design}
\label{chap:analysis}

\section{Requirements} 
Before we describe the plugin itself, the requirements need to be detailed.
There are four main parts to this:

\begin{itemize}
	\item Loading models created in CPN Tools.

	\item Annotate model with pragmatics. 
		
	\item Load sets/classes of pragmatics to add to model
	
	\item Create model specific pragmatics on the fly

\end{itemize}

\section{Test cases}

(NYI, trenger eksempler)

Simple protocol

Kao-chow


\section{Defining Pragmatics}
TODO: Detaljert info om pragmatikker og hva de er

A Pragmatic is an annotation on a model element that describes how it should be
translated into code.

Pragmatics come in three types: General, domain-specific and model-specific.

In the context of network protocols, general pragmatics include concepts
like channels, principals, and operations like opening or closing a connection,
and sending or receiving data. 

Domain-specific pragmatics are things that apply to all (or many) protocols
within a domain. An example is security protocols, where domain specific
operations are encryption and decryption, nonce generation, etc.

Model-specific pragmatics apply only to the model in which they are defined. The
WebSocket Protocol has a few examples, like masking and unmasking of frames, and
verification of received frames.

\section{THe CPN Ontology with Pragmatics}
(Obsolete ettersom Kent er på vei bort fra SADL, skal skrives om til Manchester
syntax)

\section{The CPN model type for ePNK}

TODO: Two-tiered model possibility? CPN -> Annotated CPN

\fig{CPNDefinition.pdf}{Model type diagram}{model_diagram}
Every Petri Net model created in ePNK is initially only
defined by the PNML Core Model type, but can add a PetriNetType object to extend
it with the features of another model type. Only one such object can exist for a
Petri Net. Once such an object is added, ePNK will dynamically load the related
plugin(s), and the menus for adding new objects to the net will include any new
classes that the Petri Net Type defines.

A custom Petri Net Type is made by creating an Eclipse Plugin project.
The plugin manifest then needs to be edited to define this plugin as an
extension to the org.pnml.tools.epnk.pntd extension point of ePNK.

Next, an EMF model should be created. This model should inherit
the PNML Core Model from ePNK, or any other model that already does this (like
P/T-Net and HLPNG does). The first thing this new Model Type should define is a
subclass of PetriNetType with the name of the new Model Type. This is done in
the top left corner of our diagram. This class is what will appear in the menu
to add a Petri Net Type to a new model in ePNK. 

In order to change the functionality of existing classes such as Place, Arc and
Page, we need to create subclasses of them. EMF does not support merging of
models, so we can not define new properties or relations directly on the
referenced classes of the Core Model. Instead, ePNK will check the new Petri Net
Type model for subclasses with the same name, and load these dynamically instead
of the base classes. This can be seen in our diagram, where Place, Arc,
Transition and Page are all subclassed from the classes referenced from the Core
Model. These subclasses then have references to the Pragma class. As we can see
in \figref{add_child_pragma_menu}, Pragma objects are now available as
children to Places.

\fig[0.6]{ePNK_AddChildMenu.png}{Adding Pragma as child to
Place}{add_child_pragma_menu}

The Pragma class inherits from Label, which lets it be represented visually as a
text label in the graph editor of ePNK.


\section{Importing from CPN Tools}
CPN Tools saves models in .cpn files as plain XML. Access/CPN is a set of tools
that can parse these files and represent the model with EMF classes. It has many
tools to analyse the CPN models, but only the model importer is interesting to
us.

The EMF model for CPN models that Access/CPN defines uses many of the same class
names as ePNK, making code difficult to write and read due to the need to use
fully qualified classpaths to avoid name collisions. Because of this, and the
fact that it was missing a few features like importing graphics, we early on
considered creating a new parser based on the code in Access/CPN. But it has
continually improved during development of our plugin, and is now also capable
of parsing graphics data. By using this plugin instead of rewriting it, we can
also benefit from further updates more easily.

TODO: Sjekke med repo-admin om problem med SVN, får ikke sjekket ut siste
revisjon

The conversion is very straightforward, the .cpn file is loaded with Access/CPN,
and the resulting model is then converted to the ePNK CPN type.

\section{Creating annotations}
Labels on nodes, arcs and inscriptions. 
Choose from list. Possibly write freehand with content assist. Validation, with
problem markers (Eclipse feature).

\section{Choosing Pragmatics Sets}
Where to store? Model, Project, Plugin
Model pros:
	Will need anyway for model-specific pragmatics
	Could be associated with net type as a property (like HLPNG does) or as a sub
	node somewhere
	Have URI string and version to check.
cons:
	Keeping base pragmatics up to date a problem
Namespace-based? Already required for ontology


	\subsection{Creating custom pragmatics}
	Dynamically supported in content assist 
	If ontology-based, use SADL editor
	På sikt eget verktøy
		Hva kan det settes på, hvilke attributter har det.
		Oversette til ontologi

